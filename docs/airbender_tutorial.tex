\documentclass[a4paper,10pt]{article}
\usepackage[croatian]{babel}
\usepackage{hyperref}

% tables
\usepackage{booktabs}
\usepackage{longtable}

% headers
\usepackage{fancyhdr}
\fancyhf{}
\usepackage[left=3cm,right=3cm,top=3cm,bottom=3cm]{geometry}

% toc font
\usepackage{tocloft}
%% naslov "sadrzaj"
\renewcommand{\cfttoctitlefont}{\Large\sc}
%% font stranica
\renewcommand{\cftsecpagefont}{\sc}
\renewcommand{\cftsubsecpagefont}{\sc}
%% imena poglavlja
\renewcommand{\cftsecfont}{\sc}
\renewcommand{\cftsubsecfont}{\sc}

%index
\usepackage{makeidx}
\makeindex

\lhead{\textsc{AIRbender VR}}
\rhead{\textsc{Lekcije za stvaranje igre}}
\cfoot{\thepage}

% side notes
\usepackage{todonotes}
\usepackage{xcolor}

\pagestyle{empty}
\linespread{1.3}

\usepackage{titlesec}
\titleformat*{\section}{\Large\sc}
\titleformat*{\subsection}{\large\sc}

\begin{document}

\begin{center}
\textsc{Sveučilište u Zagrebu}\\
\textsc{Fakultet organizacije i informatike}\\
\textsc{Analiza i razvoj programa}
\end{center}
\vspace*{7cm}
\begin{center}
	\textsc{\huge AIRbender VR}\\
	\textsc{\large Unreal Engine}\\Naputci za izradu\\
	\vspace*{1cm}
	Dorian Čajko \quad Ivan Huzjak \quad Denis Jocković \quad  Filip
	Novački \quad Luka Štefanić
\end{center}
\hspace*{1cm}

\begin{center}\end{center}
\vspace*{7cm}

\begin{center}
VARA\v{Z}DIN
\end{center}

\pagebreak

\pagestyle{fancy}
\tableofcontents

\pagebreak
\newgeometry{top=3cm, bottom=3cm, outer=6.5cm, inner=3cm, left=3cm,
marginparwidth=4.7cm, marginparsep=0.7cm}

\section*{Uvod}

Ovaj priručnik za programiranje projekta namijenjen je prvenstveno studentima
tehničkih područja kao nit vodilja kroz stvaranje jednog projekta. Igra koja se
stvara ovim putem temelji se na popularnoj crtanoj seriji Avatar, a ime, u
originalu \textit{Avatar, the Last Airbender}, vrlo se prikladno poklopio s
imenom kolegija, \textit{Analiza i razvoj programa}, iliti skraćeno
AIR.

\index{Uvod}
Podrazumijeva se da čitatelj ovog priručnika poznaje osnovne koncepte
programiranja te se oni ovdje neće u detalje objašnjavati. Naglasak će se
staviti na koncepte koji se upotrebljavaju u izradi igre, dakle oni vezani za
Unreal Engine, razvoj igara itd.

\marginpar{\color{teal}{Dodatna objašnjenja mogu se vidjeti izdvojena sa strane
kako bi se dodatno povezalo objašnjeno s drugim konceptima.}}
Glavni tekst sadržavat će objašnjenje postupaka koji se koriste u izradi igre.
Moguće je da se koraci malo promijene u određenim verzijama softvera koji se
koriste, no pretpostavlja se da će sve ostati slično jer se ne koriste jako
opskurni koncepti.

Na kraju priručnika nalaksi se kazalo pojmova kako bi se određeni pojam mogao
lakše pronaći ukoliko je samo spomenut negdje u tekstu, a nije iz naslova jasno
o kojem se pojmu točno radi.

\pagebreak

\section{Uvod u industriju, stvaranje igara te okruženje}

\begin{quote}
	\small
	Programiranje igara se po mnogočemu razlikuje od programiranja
	poslovnih aplikacija. Pristup i razmišljanje su drugačiji. Dok za
	poslovne aplikacije je važno usredotočiti se na podatke, u igrama je
	važno i misliti na tzv. \textit{delivery}, odnosno kako će korisnik
	doživjeti aplikaciju. Grafiku, fiziku, objekte i druga zajednička
	obilježja objedinjava \textit{game engine}. Unreal Engine je jedan od
	najpopularnijih rješenja za razvijanje igara te ćemo se u ovom
	priručniku koristiti njime.
	\marginpar{\small\color{teal}{
		U Unreal Engineu moguće je u potpunosti programirati u \texttt{C++}-u,
		no zbog kompleksnosti jezika i pristupačnosti
		\textit{blueprinta} uglavnom ćemo se baviti vizualnim
		skriptiranjem.
		}}

	Jedna od posebnost Unreal Enginea je što koristi tzv.
	\textit{blueprinte}, alat kojim se odnosi između objekata programiraju
	ne kodom, nego povlaćenjem odnosa između objekata koji su
	reprezentirani vizualno što olakšava predočavanje te dodatnu razinu
	apstrakcije od programskog jezika \texttt{C++}.

	Prije nego što će biti objašnjeni detalji o Unreal Engineu objasnit će
	se i osnove dizajna video igara (en. \textit{game design}), kako teče
	proces izrade igara te na što se sve treba pripaziti kod izrade igara.
\end{quote}

Kao dodatna motivacija za stvaranje igara je činjenica da se ta industrija u
zadnjih osam godina tržišna vrijednost igara udvostručila, a predviđa se da će
se u iduće tri godine ($2020.-2023.$) vrijednost utrostručiti. Broj aktivnih
igrača u svijetu raste velikom brzinom te se u tim podatcima porepoznaje
perspektiva te industrije. Iz tog je razloga dobro poznavanje ovog sektora, ako
ne zbog želje za radom u njoj, barem zbog opće kulture.

\subsection{Dizajn video igara}

Dizajn video igara kao proces teško je definirati. Dizajn obuhvaća sve ono što
se događa za vrijeme stvaranja igre, dakle počinje idejom i temom, nastavlja
razvitkom i na kraju stvaranje verzije igre koja se izdaje i distribuira
igračima.

\paragraph{Stvaranje ideje}

Ideja se razvija na razne načine -- može doći kroz razgovor s bliskim osobama,
može se razviti kroz \textit{brainstorming}, može doći kroz kroz bljesak
inspiracije ili na neke druge načine. Ono što je obično veći problem je doći do
jednistvenog i kvalitetnog sadržaja jer je do dana današnjeg stvoren ogroman
broj igara.

\paragraph{Skica i razrada igre}

U ovom dijelu procesa dizajna video igre kreiraju se grube skice buduće igre i
donose se razmatranja kako će izgledati pojedini element igre. Skice nisu
detaljne, ali nam uvelike olakšavaju daljnji rad u nekom od alata. Tu se
razvijaju likovi, razine, moći itd.

\paragraph{Osnovne funkcionalnosti i mehanike}

Nakon što je skica gotova dizajnira ju se mehanike - kako će objekti međusobno
reagirati, koja je njihova interakcija, što će sve likovi raditi u igri itd.
Funkcionalnosti su alati kojim se rješavaju neki problemi, npr. kretanje
glavnog lika, obrana lika od napada, umjetna inteligrncija itd.

Mehanike igre su obično neke akcije na koje igrač treba reagirati unutar igre.
Mehanike se uvelike razlikuju između različitih žanrova i to ih često čini
specifičnima. Primjeri mehanika su izazovi na \textit{bossovima}, akcije koje
tjeraju igrača na razmišljanje, zaključivanje i odlučivanje itd.

\paragraph{\textit{Game Design Document}}

Nakon što je uvodni dio napravljen, vrijeme je da se pojedini elementi malo
detaljnije razrade. Taj dokument naziva se \textit{Game Design Document}, ili
kratko -- GDD. To je detaljan dokument koji, između ostalog, sadrži:

\marginpar{\color{teal}{\small U sklopu ovih lekcija neće se raditi GDD, ali za
bilo koji ozbiljan projekt dobro je imati taj dokument kao zamjenu za
dokumentaciju kako bi se olakšalo snalaženje u projektu i kodu.}}

\begin{itemize}
	\item naziv igre
	\item sažetak igre
	\item funkcionalnosti
	\item mehaniku
	\item opis likova
	\item dizajn razina
	\item zvučne efekte
\end{itemize}

Cilj dokumenta je olakšati svima razvoj igre tako da lakše zajedno surađuju. U
njemu su opisane sve glavne komponente

\subsection{VR}

Igra-projekt kojeg će ovaj priručnik opisati bit će implementiran za VR.
\marginpar{\color{teal}{\small VR može vrlo lako učiniti neiskusnog igrača
omamljenog, odnosno može osjećati glavobolju, vrtoglavicu i slične simptome
ukoliko nije naviknut na virtualnu stvarnost. Postoje mnoge tehnike kako se to
može ublažiti. U ovom projektu će se pokušati voditi računa o tome koliko god
je moguće.}} VR je skraćenica na engleskom od \textit{virtual reality} što
znači da se imitira stvarnost u virtualnom okruženju. Cilj VR-a je korisniku
stvoriti osjećaj kao da je u stvarnom svijetu podražujući više osjetila.
Glavni uređaj koji je svojevrsno obilježje VR koncepta su naočale, odnosno
\textit{headset} koji prekriva cijelo vidno polje i korisniku omogućuje da vidi
sliku kao realnost.  Ovaj projekt koristit će i kontrolere koji omogućuju
korisniku pokretanje.

Pomoću kontrolera moguće je pratiti pokrete ruku. U nekim igrama to je već
iskorišteno za precizno bacanje projektila, predmeta, uzimanje predmeta itd., a
u našem projektu to će biti glavni okidači za usmjereno bacanje moći.
moći glavnog lika.


\subsection{Unreal Engine}

Kako bi se instalirao Unreal Engine, potrebno je otići na mrežnu stranicu
stranicu \url{unrealengine.com}. U gornjem desnom kutu su opcije \texttt{Sign
in} te \texttt{Download}. Prijava je potrebna za pokretanje Unreal Enginea tako
da se korisnik mora registrirati prije ili kasnije.

\marginpar{\color{teal}{\small Instalacija za Linux ponešto je složenija te je potrebno
kompajlirati cijeli projekt. To uzima poprilično vremena i resursa, a sadrži i
nešto više koraka koji se ovdje neće opisati jer se orijentiramo na Windows
operacijske sustave}}

Kod odabira licence potrebno je pripaziti koja se odabire. \textit{Publishing
license} ona je koja se odabire ukoliko se proizvod namjerava prodati, a
\textit{Creators license} ukoliko se namjerava raditi nemonetiziran rad.
Studenti su navedeni u obje kategorije jer se \textit{engine} ne mora plaćati
ako se koristi za projekt koji još ne stvara profit.

Dalje je potrebno registrirati se te slijediti uputstva kod instalacije. Unreal
Engine radi na operacijskim sustavima Linux te Windows.

\subsection{Kreiranje projekta}

\subsection{Zaključak i otvorena pitanja}

Ovo je bio uvod i treba nas pripremiti za sve ono što dolazi u idućoj lekciji,
a to je postavljanje prvih objekata u svije.

Ono čime se ovdje može dodatno obogatiti projekt su:
\begin{itemize}
	\item otvori prostor za dodatna poboljšanja prvi primjer
	\item prijedlog drugog poboljšanja
	\item za učinak koji želimo treći primjer
\end{itemize}



\pagebreak
\section{Kreiranje i kretanje (teleportacija) glavnog lika}
\index{Vektor gledanja}
\index{Teleportacija}

\pagebreak
\section{Detekcija pokreta kontrolera}

\pagebreak
\section{Stvaranje meta}

\pagebreak
\section{Napadačke moći}

\pagebreak
\section{Stvaranje prijetnji za igrača i prikaz zdravlja (HP)}

\pagebreak
\section{Obrambeni mehanizmi za igrača}

\pagebreak
\section{Bodovanje i prikaz bodova}

\pagebreak
\section{Izrada krajolika nivoa}

\pagebreak
\section{Izbornici}

\pagebreak
\section{"Umjetna inteligencija" -- protivnik se kreće}

\pagebreak
\section{Vizualni efekti}

\pagebreak
\section{Zvuk}

\printindex
\end{document}
